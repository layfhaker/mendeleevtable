% =========================================
% ШАБЛОН ДОКУМЕНТА ДЛЯ ХИМИЧЕСКОГО ЭЛЕМЕНТА
% =========================================
\documentclass[12pt,a4paper]{article}

% Пакеты
\usepackage[utf8]{inputenc}
\usepackage[russian]{babel}
\usepackage[T2A]{fontenc}
\usepackage{geometry}
\usepackage{graphicx}
\usepackage{xcolor}
\usepackage{array}
\usepackage{longtable}
\usepackage{booktabs}
\usepackage{fancyhdr}
\usepackage{lastpage}
\usepackage{hyperref}
\usepackage[version=4]{mhchem} % Для химических формул

% Настройка страницы
\geometry{
    a4paper,
    left=20mm,
    right=20mm,
    top=25mm,
    bottom=25mm
}

% Цвета
\definecolor{primaryBlue}{RGB}{33,150,243}
\definecolor{secondaryGreen}{RGB}{76,175,80}
\definecolor{accentOrange}{RGB}{255,152,0}
\definecolor{accentPurple}{RGB}{156,39,176}
\definecolor{lightGray}{RGB}{245,245,245}

% Настройка гиперссылок
\hypersetup{
    colorlinks=true,
    linkcolor=primaryBlue,
    urlcolor=primaryBlue,
    pdftitle={Химический элемент: {{ELEMENT_NAME}}},
    pdfauthor={Интерактивная Таблица Менделеева}
}

% Колонтитулы
\pagestyle{fancy}
\fancyhf{}
\fancyhead[L]{\small\textcolor{gray}{Химический элемент: {{ELEMENT_NAME}}}}
\fancyhead[R]{\small\textcolor{gray}{\thepage\ из \pageref{LastPage}}}
\renewcommand{\headrulewidth}{0.4pt}

% Начало документа
\begin{document}

% =========================================
% ТИТУЛЬНАЯ ИНФОРМАЦИЯ
% =========================================
\begin{center}
    \vspace*{1cm}
    {\Huge\bfseries\color{primaryBlue} {{ELEMENT_NAME}} ({{ELEMENT_SYMBOL}})}\\[0.5cm]
    \rule{\textwidth}{0.4pt}\\[0.3cm]
    {\large Химический элемент таблицы Менделеева}\\[0.3cm]
    \rule{\textwidth}{0.4pt}\\[1cm]
    
    {\small Документ сгенерирован: {{GENERATION_DATE}}}
\end{center}

\vspace{1cm}

% =========================================
% БАЗОВАЯ ИНФОРМАЦИЯ
% =========================================
\section*{\color{primaryBlue}Базовая информация}

\begin{table}[h!]
\centering
\rowcolors{2}{lightGray}{white}
\begin{tabular}{>{\bfseries}p{6cm}p{8cm}}
\toprule
\rowcolor{primaryBlue}
\textcolor{white}{Свойство} & \textcolor{white}{Значение} \\
\midrule
Атомный номер & {{ATOMIC_NUMBER}} \\
Атомная масса & {{ATOMIC_MASS}} \\
Период & {{PERIOD}} \\
Группа & {{GROUP}} \\
Блок & {{BLOCK}} \\
Категория & {{CATEGORY}} \\
Электронная конфигурация & \texttt{{{ELECTRON_CONFIG}}} \\
Электроотрицательность & {{ELECTRONEGATIVITY}} \\
\bottomrule
\end{tabular}
\end{table}

% =========================================
% ФИЗИЧЕСКИЕ СВОЙСТВА
% =========================================
\section*{\color{secondaryGreen}Физические свойства}

\begin{table}[h!]
\centering
\rowcolors{2}{lightGray}{white}
\begin{tabular}{>{\bfseries}p{6cm}p{8cm}}
\toprule
\rowcolor{secondaryGreen}
\textcolor{white}{Свойство} & \textcolor{white}{Значение} \\
\midrule
Плотность & {{DENSITY}} \\
Температура плавления & {{MELTING_POINT}} \\
Температура кипения & {{BOILING_POINT}} \\
Состояние (20°C) & {{STATE}} \\
Цвет & {{COLOR}} \\
\bottomrule
\end{tabular}
\end{table}

% =========================================
% ИСТОРИЯ И ПРИМЕНЕНИЕ
% =========================================
\section*{\color{accentOrange}История и применение}

\begin{table}[h!]
\centering
\rowcolors{2}{lightGray}{white}
\begin{tabular}{>{\bfseries}p{6cm}p{8cm}}
\toprule
\rowcolor{accentOrange}
\textcolor{white}{Свойство} & \textcolor{white}{Значение} \\
\midrule
Год открытия & {{DISCOVERY_YEAR}} \\
Первооткрыватель & {{DISCOVERER}} \\
Происхождение названия & {{NAME_ORIGIN}} \\
Области применения & {{APPLICATIONS}} \\
\bottomrule
\end{tabular}
\end{table}

% =========================================
% ИНТЕРЕСНЫЕ ФАКТЫ
% =========================================
\section*{\color{primaryBlue}Интересные факты}

{{FACTS}}

% =========================================
% СОЕДИНЕНИЯ ИЗ ТАБЛИЦЫ РАСТВОРИМОСТИ
% =========================================
\newpage

\section*{\color{accentPurple}Соединения {{ELEMENT_SYMBOL}} из таблицы растворимости}

\begin{longtable}{p{3cm}p{3.5cm}p{2.5cm}p{5cm}}
\toprule
\rowcolor{accentPurple}
\textcolor{white}{\bfseries Формула} & 
\textcolor{white}{\bfseries Растворимость} & 
\textcolor{white}{\bfseries Цвет} & 
\textcolor{white}{\bfseries Реакция} \\
\midrule
\endfirsthead

\multicolumn{4}{c}{\textit{Продолжение таблицы}}\\
\toprule
\rowcolor{accentPurple}
\textcolor{white}{\bfseries Формула} & 
\textcolor{white}{\bfseries Растворимость} & 
\textcolor{white}{\bfseries Цвет} & 
\textcolor{white}{\bfseries Реакция} \\
\midrule
\endhead

\midrule
\multicolumn{4}{r}{\textit{Продолжение на следующей странице}} \\
\endfoot

\bottomrule
\endlastfoot

{{COMPOUNDS_TABLE}}

\end{longtable}

% =========================================
% ПОДВАЛ ДОКУМЕНТА
% =========================================
\vfill
\begin{center}
\small
\rule{\textwidth}{0.2pt}\\[0.2cm]
\textcolor{gray}{Документ создан с помощью Интерактивной Таблицы Менделеева}\\
\textcolor{gray}{\url{https://github.com/layfhaker/mendeleevtable}}
\end{center}

\end{document}
